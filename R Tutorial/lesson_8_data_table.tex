\PassOptionsToPackage{unicode=true}{hyperref} % options for packages loaded elsewhere
\PassOptionsToPackage{hyphens}{url}
%
\documentclass[ignorenonframetext,]{beamer}
\usepackage{pgfpages}
\setbeamertemplate{caption}[numbered]
\setbeamertemplate{caption label separator}{: }
\setbeamercolor{caption name}{fg=normal text.fg}
\beamertemplatenavigationsymbolsempty
\usepackage{lmodern}
\usepackage{amssymb,amsmath}
\usepackage{ifxetex,ifluatex}
\usepackage{fixltx2e} % provides \textsubscript
\ifnum 0\ifxetex 1\fi\ifluatex 1\fi=0 % if pdftex
  \usepackage[T1]{fontenc}
  \usepackage[utf8]{inputenc}
  \usepackage{textcomp} % provides euro and other symbols
\else % if luatex or xelatex
  \usepackage{unicode-math}
  \defaultfontfeatures{Ligatures=TeX,Scale=MatchLowercase}
\fi
\usetheme[]{Singapore}
% use upquote if available, for straight quotes in verbatim environments
\IfFileExists{upquote.sty}{\usepackage{upquote}}{}
% use microtype if available
\IfFileExists{microtype.sty}{%
\usepackage[]{microtype}
\UseMicrotypeSet[protrusion]{basicmath} % disable protrusion for tt fonts
}{}
\IfFileExists{parskip.sty}{%
\usepackage{parskip}
}{% else
\setlength{\parindent}{0pt}
\setlength{\parskip}{6pt plus 2pt minus 1pt}
}
\usepackage{hyperref}
\hypersetup{
            pdftitle={The Case for data.table},
            pdfauthor={Eric Karsten},
            pdfborder={0 0 0},
            breaklinks=true}
\urlstyle{same}  % don't use monospace font for urls
\newif\ifbibliography
\usepackage{color}
\usepackage{fancyvrb}
\newcommand{\VerbBar}{|}
\newcommand{\VERB}{\Verb[commandchars=\\\{\}]}
\DefineVerbatimEnvironment{Highlighting}{Verbatim}{commandchars=\\\{\}}
% Add ',fontsize=\small' for more characters per line
\usepackage{framed}
\definecolor{shadecolor}{RGB}{248,248,248}
\newenvironment{Shaded}{\begin{snugshade}}{\end{snugshade}}
\newcommand{\AlertTok}[1]{\textcolor[rgb]{0.94,0.16,0.16}{#1}}
\newcommand{\AnnotationTok}[1]{\textcolor[rgb]{0.56,0.35,0.01}{\textbf{\textit{#1}}}}
\newcommand{\AttributeTok}[1]{\textcolor[rgb]{0.77,0.63,0.00}{#1}}
\newcommand{\BaseNTok}[1]{\textcolor[rgb]{0.00,0.00,0.81}{#1}}
\newcommand{\BuiltInTok}[1]{#1}
\newcommand{\CharTok}[1]{\textcolor[rgb]{0.31,0.60,0.02}{#1}}
\newcommand{\CommentTok}[1]{\textcolor[rgb]{0.56,0.35,0.01}{\textit{#1}}}
\newcommand{\CommentVarTok}[1]{\textcolor[rgb]{0.56,0.35,0.01}{\textbf{\textit{#1}}}}
\newcommand{\ConstantTok}[1]{\textcolor[rgb]{0.00,0.00,0.00}{#1}}
\newcommand{\ControlFlowTok}[1]{\textcolor[rgb]{0.13,0.29,0.53}{\textbf{#1}}}
\newcommand{\DataTypeTok}[1]{\textcolor[rgb]{0.13,0.29,0.53}{#1}}
\newcommand{\DecValTok}[1]{\textcolor[rgb]{0.00,0.00,0.81}{#1}}
\newcommand{\DocumentationTok}[1]{\textcolor[rgb]{0.56,0.35,0.01}{\textbf{\textit{#1}}}}
\newcommand{\ErrorTok}[1]{\textcolor[rgb]{0.64,0.00,0.00}{\textbf{#1}}}
\newcommand{\ExtensionTok}[1]{#1}
\newcommand{\FloatTok}[1]{\textcolor[rgb]{0.00,0.00,0.81}{#1}}
\newcommand{\FunctionTok}[1]{\textcolor[rgb]{0.00,0.00,0.00}{#1}}
\newcommand{\ImportTok}[1]{#1}
\newcommand{\InformationTok}[1]{\textcolor[rgb]{0.56,0.35,0.01}{\textbf{\textit{#1}}}}
\newcommand{\KeywordTok}[1]{\textcolor[rgb]{0.13,0.29,0.53}{\textbf{#1}}}
\newcommand{\NormalTok}[1]{#1}
\newcommand{\OperatorTok}[1]{\textcolor[rgb]{0.81,0.36,0.00}{\textbf{#1}}}
\newcommand{\OtherTok}[1]{\textcolor[rgb]{0.56,0.35,0.01}{#1}}
\newcommand{\PreprocessorTok}[1]{\textcolor[rgb]{0.56,0.35,0.01}{\textit{#1}}}
\newcommand{\RegionMarkerTok}[1]{#1}
\newcommand{\SpecialCharTok}[1]{\textcolor[rgb]{0.00,0.00,0.00}{#1}}
\newcommand{\SpecialStringTok}[1]{\textcolor[rgb]{0.31,0.60,0.02}{#1}}
\newcommand{\StringTok}[1]{\textcolor[rgb]{0.31,0.60,0.02}{#1}}
\newcommand{\VariableTok}[1]{\textcolor[rgb]{0.00,0.00,0.00}{#1}}
\newcommand{\VerbatimStringTok}[1]{\textcolor[rgb]{0.31,0.60,0.02}{#1}}
\newcommand{\WarningTok}[1]{\textcolor[rgb]{0.56,0.35,0.01}{\textbf{\textit{#1}}}}
% Prevent slide breaks in the middle of a paragraph:
\widowpenalties 1 10000
\raggedbottom
\setbeamertemplate{part page}{
\centering
\begin{beamercolorbox}[sep=16pt,center]{part title}
  \usebeamerfont{part title}\insertpart\par
\end{beamercolorbox}
}
\setbeamertemplate{section page}{
\centering
\begin{beamercolorbox}[sep=12pt,center]{part title}
  \usebeamerfont{section title}\insertsection\par
\end{beamercolorbox}
}
\setbeamertemplate{subsection page}{
\centering
\begin{beamercolorbox}[sep=8pt,center]{part title}
  \usebeamerfont{subsection title}\insertsubsection\par
\end{beamercolorbox}
}
\AtBeginPart{
  \frame{\partpage}
}
\AtBeginSection{
  \ifbibliography
  \else
    \frame{\sectionpage}
  \fi
}
\AtBeginSubsection{
  \frame{\subsectionpage}
}
\setlength{\emergencystretch}{3em}  % prevent overfull lines
\providecommand{\tightlist}{%
  \setlength{\itemsep}{0pt}\setlength{\parskip}{0pt}}
\setcounter{secnumdepth}{0}

% set default figure placement to htbp
\makeatletter
\def\fps@figure{htbp}
\makeatother


\title{The Case for data.table}
\author{Eric Karsten}
\date{25 July, 2018}

\begin{document}
\frame{\titlepage}

\hypertarget{the-options}{%
\section{The Options}\label{the-options}}

\begin{frame}[fragile]{Base R}
\protect\hypertarget{base-r}{}

Pros:

\begin{itemize}
\tightlist
\item
  It's the first thing you learn
\end{itemize}

Cons:

\begin{itemize}
\tightlist
\item
  Code is unreadable
\item
  Iterative processes require for loops (unclear) or
\item
  calling the \texttt{apply} function (slow)
\end{itemize}

\end{frame}

\begin{frame}{dplyr and the tidyverse}
\protect\hypertarget{dplyr-and-the-tidyverse}{}

Pros:

\begin{itemize}
\tightlist
\item
  The syntax is very readable
\item
  Major speed improvements over base R
\item
  More complex aggregations possible
\end{itemize}

Cons:

\begin{itemize}
\tightlist
\item
  You will eventually come across aggregations you can't do
\item
  It's a little slow when you start doing big stuff
\item
  You can't poke around the backend code easily (it's in C++)
\end{itemize}

\end{frame}

\begin{frame}{data.table}
\protect\hypertarget{data.table}{}

Pros:

\begin{itemize}
\tightlist
\item
  Almost every aggregation you can think of is possible
\item
  Some speed improvements over dplyr (task-dependent)
\item
  Significanlty more efficient use of RAM than dplyr or base R
\end{itemize}

Cons:

\begin{itemize}
\tightlist
\item
  Sytax is hard to read
\item
  You have to know what you are doing to get the most efficiency out of
  it (this presentation will help with that!)
\item
  You can't poke around the backend code easily (it's in C++)
\end{itemize}

\end{frame}

\hypertarget{simple-example}{%
\section{Simple Example}\label{simple-example}}

\begin{frame}[fragile]{Our Data}
\protect\hypertarget{our-data}{}

This is a super simple example so that you can see all the parts.

\begin{Shaded}
\begin{Highlighting}[]
\CommentTok{# you may need to `install.packages("data.table")`}

\KeywordTok{library}\NormalTok{(data.table)}

\NormalTok{df <-}
\StringTok{  }\KeywordTok{data.table}\NormalTok{(}
    \DataTypeTok{Animal =} \KeywordTok{c}\NormalTok{(}\StringTok{"Dog"}\NormalTok{, }\StringTok{"Cat"}\NormalTok{, }\StringTok{"Dog"}\NormalTok{,}
               \StringTok{"Raven"}\NormalTok{, }\StringTok{"Cat"}\NormalTok{), }
    \DataTypeTok{Weight =} \KeywordTok{c}\NormalTok{(}\DecValTok{100}\NormalTok{, }\DecValTok{40}\NormalTok{, }\DecValTok{80}\NormalTok{, }\DecValTok{16}\NormalTok{, }\DecValTok{50}\NormalTok{),}
    \DataTypeTok{Height =} \KeywordTok{c}\NormalTok{(}\DecValTok{23}\NormalTok{, }\DecValTok{18}\NormalTok{, }\DecValTok{40}\NormalTok{, }\DecValTok{3}\NormalTok{, }\DecValTok{16}\NormalTok{),}
    \DataTypeTok{Family =} \KeywordTok{c}\NormalTok{(}\StringTok{"Mammal"}\NormalTok{, }\StringTok{"Mammal"}\NormalTok{, }\StringTok{"Mammal"}\NormalTok{,}
               \StringTok{"Bird"}\NormalTok{, }\StringTok{"Mammal"}\NormalTok{))}
\end{Highlighting}
\end{Shaded}

\end{frame}

\begin{frame}{The Syntax}
\protect\hypertarget{the-syntax}{}

There are three parts to a data.table command:

\begin{enumerate}
\tightlist
\item
  The rows to be returned
\item
  The columns to be returned
\item
  The grouping by which to perform the operations
\end{enumerate}

They are separated by commas, but commas need only be included when a
subsequent field is occupied.

\end{frame}

\begin{frame}[fragile]{Some nice easy subsetting}
\protect\hypertarget{some-nice-easy-subsetting}{}

Let's just select the dogs.

\begin{Shaded}
\begin{Highlighting}[]
\NormalTok{df[Animal }\OperatorTok{==}\StringTok{ "Dog"}\NormalTok{]}
\end{Highlighting}
\end{Shaded}

\begin{verbatim}
##    Animal Weight Height Family
## 1:    Dog    100     23 Mammal
## 2:    Dog     80     40 Mammal
\end{verbatim}

Let's return the height and weight of the mammals that weigh more than
35 lbs

\begin{Shaded}
\begin{Highlighting}[]
\NormalTok{df[Weight }\OperatorTok{>}\StringTok{ }\DecValTok{35}\NormalTok{, .(Animal, Height, Weight)]}
\end{Highlighting}
\end{Shaded}

\begin{verbatim}
##    Animal Height Weight
## 1:    Dog     23    100
## 2:    Cat     18     40
## 3:    Dog     40     80
## 4:    Cat     16     50
\end{verbatim}

\end{frame}

\begin{frame}[fragile]{Using \texttt{:=}}
\protect\hypertarget{using}{}

Often times, we want to add new calculated columns to a dataframe. The
correct way to do this in data.table is with a \texttt{:=}. Because this
operation works by updating the previous object, we need to make a copy
of it in order to avoid modifying our original data frame.

The following are equivalent for computing BMI (weight/height)
\footnote{BMI
should technically be weight in Kg divided by height in m, but I'm not going to
worry about the conversions}

\begin{Shaded}
\begin{Highlighting}[]
\NormalTok{df2 <-}\StringTok{ }\KeywordTok{copy}\NormalTok{(df)[, BMI }\OperatorTok{:}\ErrorTok{=}\StringTok{ }\NormalTok{Weight}\OperatorTok{/}\NormalTok{Height]}
\NormalTok{df2 <-}\StringTok{ }\KeywordTok{copy}\NormalTok{(df)[, }\StringTok{`}\DataTypeTok{:=}\StringTok{`}\NormalTok{(}\DataTypeTok{BMI =}\NormalTok{ Weight}\OperatorTok{/}\NormalTok{Height)]}
\end{Highlighting}
\end{Shaded}

\begin{verbatim}
##    Animal Weight Height Family      BMI
## 1:    Dog    100     23 Mammal 4.347826
## 2:    Cat     40     18 Mammal 2.222222
## 3:    Dog     80     40 Mammal 2.000000
## 4:  Raven     16      3   Bird 5.333333
## 5:    Cat     50     16 Mammal 3.125000
\end{verbatim}

\end{frame}

\begin{frame}[fragile]

We could also have done this without efficient updating. This is much
slower on big datasets and should not be done.

\begin{Shaded}
\begin{Highlighting}[]
\NormalTok{df[, .(Animal, Weight, Height, Family,}
       \DataTypeTok{BMI =}\NormalTok{ Weight}\OperatorTok{/}\NormalTok{Height)]}
\end{Highlighting}
\end{Shaded}

\begin{verbatim}
##    Animal Weight Height Family      BMI
## 1:    Dog    100     23 Mammal 4.347826
## 2:    Cat     40     18 Mammal 2.222222
## 3:    Dog     80     40 Mammal 2.000000
## 4:  Raven     16      3   Bird 5.333333
## 5:    Cat     50     16 Mammal 3.125000
\end{verbatim}

\end{frame}

\begin{frame}[fragile]

Selective updating is the reason why \texttt{:=} is so useful. Let's say
we wanted to change the hieght of all ravens to be 5. We don't want to
rewrite the whole table, so we simply call

\begin{Shaded}
\begin{Highlighting}[]
\NormalTok{df[Animal }\OperatorTok{==}\StringTok{ "Raven"}\NormalTok{, Height }\OperatorTok{:}\ErrorTok{=}\StringTok{ }\DecValTok{5}\NormalTok{]}
\end{Highlighting}
\end{Shaded}

\begin{verbatim}
##    Animal Weight Height Family
## 1:    Dog    100     23 Mammal
## 2:    Cat     40     18 Mammal
## 3:    Dog     80     40 Mammal
## 4:  Raven     16      5   Bird
## 5:    Cat     50     16 Mammal
\end{verbatim}

\end{frame}

\begin{frame}[fragile]{Grouped operations}
\protect\hypertarget{grouped-operations}{}

Let's say we want the average weight by animal. Here are two ways to do
it with different results.

\begin{Shaded}
\begin{Highlighting}[]
\NormalTok{df[, .(}\DataTypeTok{mean_wt =} \KeywordTok{mean}\NormalTok{(Weight)), by =}\StringTok{ }\NormalTok{.(Animal)]}
\end{Highlighting}
\end{Shaded}

\begin{verbatim}
##    Animal mean_wt
## 1:    Dog      90
## 2:    Cat      45
## 3:  Raven      16
\end{verbatim}

\begin{Shaded}
\begin{Highlighting}[]
\NormalTok{df2 <-}\StringTok{ }\KeywordTok{copy}\NormalTok{(df)[, mean_wt }\OperatorTok{:}\ErrorTok{=}\StringTok{ }\KeywordTok{mean}\NormalTok{(Weight), by =}\StringTok{ }\NormalTok{Animal]}
\end{Highlighting}
\end{Shaded}

\begin{verbatim}
##    Animal Weight Height Family mean_wt
## 1:    Dog    100     23 Mammal      90
## 2:    Cat     40     18 Mammal      45
## 3:    Dog     80     40 Mammal      90
## 4:  Raven     16      5   Bird      16
## 5:    Cat     50     16 Mammal      45
\end{verbatim}

\end{frame}

\begin{frame}[fragile]{Chaining multiple operations in sequence}
\protect\hypertarget{chaining-multiple-operations-in-sequence}{}

Oftentimes, you want to do several operations in sequence. Naturally,
there is syntax that reflects this workflow. Here we are computing BMI
and then making a table of average BMI by Animal.

\begin{Shaded}
\begin{Highlighting}[]
\NormalTok{df[, }\StringTok{`}\DataTypeTok{:=}\StringTok{`}\NormalTok{(}\DataTypeTok{BMI =}\NormalTok{ Weight}\OperatorTok{/}\NormalTok{Height)][}
\NormalTok{  , .(}\DataTypeTok{avg_BMI =} \KeywordTok{mean}\NormalTok{(BMI)), by =}\StringTok{ }\NormalTok{Animal]}
\end{Highlighting}
\end{Shaded}

\begin{verbatim}
##    Animal  avg_BMI
## 1:    Dog 3.173913
## 2:    Cat 2.673611
## 3:  Raven 3.200000
\end{verbatim}

It should be noted that this example could be done in one line, so
chaining was not necessary.

\end{frame}

\begin{frame}[fragile]{Using \texttt{.SD} tricks}
\protect\hypertarget{using-.sd-tricks}{}

There are times when we want to perform more complex operations on our
grouped data than applying a simple built-in function. One option is to
write a function of our own, but this isn't always the clearest. The
better option is to manipulate the grouped data.table using standard
data.table operations!

The way we do this is by calling the \texttt{.SD} object (which is just
a data.table of our selected data) and performing operations on it.

EX: We want to add a column for the height of the heaviest individual
within each Animal species.

To do this, we need to first compute within each animal table what the
biggest weight is, then, we need to return the height of the animal with
that weight and assign it to the BigHeight variable for that animal
group.

\end{frame}

\begin{frame}[fragile]

\begin{Shaded}
\begin{Highlighting}[]
\NormalTok{df[, BigHeight }\OperatorTok{:}\ErrorTok{=}\StringTok{ }\NormalTok{.SD[, .(}\DataTypeTok{BiggestWeight =} \KeywordTok{max}\NormalTok{(Weight),}
\NormalTok{                           Weight,}
\NormalTok{                           Height)][}
\NormalTok{                       Weight }\OperatorTok{==}\StringTok{ }\NormalTok{BiggestWeight, Height]}
\NormalTok{   , by =}\StringTok{ }\NormalTok{Animal]}
\end{Highlighting}
\end{Shaded}

\begin{verbatim}
##    Animal Weight Height Family      BMI BigHeight
## 1:    Dog    100     23 Mammal 4.347826        23
## 2:    Cat     40     18 Mammal 2.222222        16
## 3:    Dog     80     40 Mammal 2.000000        23
## 4:  Raven     16      5   Bird 3.200000         5
## 5:    Cat     50     16 Mammal 3.125000        16
\end{verbatim}

I challenge you to do this in either dplyr or base R in a cleaner or
faster way!

\end{frame}

\hypertarget{speed-tricks}{%
\section{Speed Tricks}\label{speed-tricks}}

\begin{frame}{Why is it so fast?}
\protect\hypertarget{why-is-it-so-fast}{}

Both data.table and dplyr are executed in C++. This allows the
developers to use some really efficient indexing to perform their
operations.

The advantage data.table has over dplyr in the speed department is that
it is able to update the data in RAM, it doesn't need to constantly be
rewriting it. This saves a ton of time if you are doing thousands of
small operations in a row. It also allows the program to run without
eating up all your RAM!

\end{frame}

\begin{frame}[fragile]{How to maximize speed}
\protect\hypertarget{how-to-maximize-speed}{}

Always use \texttt{:=} when possible. You never want to rewrite your
data.table if you don't have to.

To delete a column quickly, just use \texttt{:=\ NULL} to get rid of it.

If you are doing several things on the same group, don't chain the
operations together, instead do them all at once.

Merging isn't covered in this talk, but data.table can do really fast
merging.

\end{frame}

\hypertarget{proof}{%
\section{Proof}\label{proof}}

\begin{frame}[fragile]{The task}
\protect\hypertarget{the-task}{}

We are going to use the microbenchmark library to find the average BMI
of our animals, after adding 7 lbs to all the rabbits but to slow it
down, we are going to randomly generate a ton of data.

\begin{Shaded}
\begin{Highlighting}[]
\KeywordTok{library}\NormalTok{(microbenchmark)}
\KeywordTok{library}\NormalTok{(dplyr)}
\NormalTok{n <-}\StringTok{ }\FloatTok{1e5}
\NormalTok{dt_test <-}
\StringTok{  }\KeywordTok{data.table}\NormalTok{(}\DataTypeTok{Animal =} \KeywordTok{sample}\NormalTok{(}\KeywordTok{c}\NormalTok{(}\StringTok{"Rabbit"}\NormalTok{, }\StringTok{"Chicken"}\NormalTok{,}
                               \StringTok{"Zebra"}\NormalTok{, }\StringTok{"Snake"}\NormalTok{,}
                               \StringTok{"Giraffe"}\NormalTok{, }\StringTok{"Eagle"}\NormalTok{),}
\NormalTok{                             n, }\DataTypeTok{replace =}\NormalTok{ T),}
             \DataTypeTok{Height =} \KeywordTok{runif}\NormalTok{(n, }\DecValTok{3}\NormalTok{, }\DecValTok{7}\NormalTok{),}
             \DataTypeTok{Weight =} \KeywordTok{runif}\NormalTok{(n, }\DecValTok{33}\NormalTok{, }\DecValTok{47}\NormalTok{))}
\NormalTok{df_test <-}\StringTok{ }\KeywordTok{as.data.frame}\NormalTok{(dt_test)}
\NormalTok{tb_test <-}\StringTok{ }\KeywordTok{as_tibble}\NormalTok{(dt_test)}
\end{Highlighting}
\end{Shaded}

\end{frame}

\begin{frame}[fragile]{The Benchmark Setup}
\protect\hypertarget{the-benchmark-setup}{}

\begin{Shaded}
\begin{Highlighting}[]
\NormalTok{data_table_BMI <-}\StringTok{ }\ControlFlowTok{function}\NormalTok{(df) \{}
\NormalTok{  df[Animal }\OperatorTok{==}\StringTok{ "Rabbit"}\NormalTok{, Weight }\OperatorTok{:}\ErrorTok{=}\StringTok{ }\NormalTok{Weight }\OperatorTok{+}\StringTok{ }\DecValTok{7}\NormalTok{][}
\NormalTok{    , .(}\DataTypeTok{avg_BMI=} \KeywordTok{mean}\NormalTok{(Weight}\OperatorTok{/}\NormalTok{Height)), by =}\StringTok{ }\NormalTok{Animal]\}}
\NormalTok{dplyr_BMI <-}\StringTok{ }\ControlFlowTok{function}\NormalTok{(df) \{}
\NormalTok{  df }\OperatorTok
\StringTok{    }\KeywordTok{mutate}\NormalTok{(}\DataTypeTok{Weight =} \KeywordTok{if_else}\NormalTok{(Animal }\OperatorTok{==}\StringTok{ "Rabbit"}\NormalTok{,}
\NormalTok{                            Weight }\OperatorTok{+}\StringTok{ }\DecValTok{7}\NormalTok{,Weight)) }\OperatorTok
\StringTok{    }\KeywordTok{group_by}\NormalTok{(Animal) }\OperatorTok
\StringTok{    }\KeywordTok{summarise}\NormalTok{(}\DataTypeTok{avg_BMI =} \KeywordTok{mean}\NormalTok{(Weight}\OperatorTok{/}\NormalTok{Height))\}}
\NormalTok{base_R_BMI <-}\StringTok{ }\ControlFlowTok{function}\NormalTok{(df) \{}
\NormalTok{  df}\OperatorTok{$}\NormalTok{Weight =}\StringTok{ }\KeywordTok{ifelse}\NormalTok{(df}\OperatorTok{$}\NormalTok{Animal }\OperatorTok{==}\StringTok{ "Rabbit"}\NormalTok{,}
\NormalTok{                     df}\OperatorTok{$}\NormalTok{Weight }\OperatorTok{+}\StringTok{ }\DecValTok{7}\NormalTok{,df}\OperatorTok{$}\NormalTok{Weight)}
  \KeywordTok{aggregate}\NormalTok{(df}\OperatorTok{$}\NormalTok{Weight}\OperatorTok{/}\NormalTok{df}\OperatorTok{$}\NormalTok{Height,}
            \DataTypeTok{by=}\KeywordTok{list}\NormalTok{(}\DataTypeTok{Animal=}\NormalTok{df}\OperatorTok{$}\NormalTok{Animal),}
            \DataTypeTok{FUN=}\NormalTok{mean)\}}
\end{Highlighting}
\end{Shaded}

\end{frame}

\begin{frame}[fragile]{Benchmark Results}
\protect\hypertarget{benchmark-results}{}

\begin{Shaded}
\begin{Highlighting}[]
\NormalTok{res <-}
\StringTok{  }\KeywordTok{microbenchmark}\NormalTok{(}
    \KeywordTok{data_table_BMI}\NormalTok{(dt_test),}
    \KeywordTok{dplyr_BMI}\NormalTok{(tb_test),}
    \KeywordTok{base_R_BMI}\NormalTok{(df_test))}
\end{Highlighting}
\end{Shaded}

\begin{verbatim}
## # A tibble: 3 x 4
##   Method      Mean Median    SD
##   <chr>      <dbl>  <dbl> <dbl>
## 1 Base R     67.5   67.0   5.08
## 2 data.table  7.42   7.19  0.93
## 3 dplyr      12.3   10.6   6.3
\end{verbatim}

\end{frame}

\hypertarget{concluding-thoughts}{%
\section{Concluding Thoughts}\label{concluding-thoughts}}

\begin{frame}{A Handy Guide}
\protect\hypertarget{a-handy-guide}{}

\begin{itemize}
\tightlist
\item
  If you data is less than 100 rows, skip the computer and do it with
  pencil and paper, it really doesn't even matter
\item
  Less than 1000 and everything will feel fast
\item
  Less than 1,000,000 and dplyr should be alright
\item
  Less than 10,000,000 and data.table will work on your laptop
\item
  Otherwise, it's time to go shopping for server space
\end{itemize}

\end{frame}

\begin{frame}{Thank You}
\protect\hypertarget{thank-you}{}

A much more comprehensive data.table guide is provided here:
\url{https://cran.r-project.org/web/packages/data.table/vignettes/datatable-intro.html}

\end{frame}

\end{document}
